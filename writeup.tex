\documentclass{article}
\begin{document}
Weber's law says the gravitational force of object $i$ on object $j$ is: $$F_{ij}=\frac{Gm_im_j}{r^2}\left[1-\frac{\xi}{2c^2}\dot{r}^2+\frac{\xi}{c^2}r\ddot{r}\right],$$ where $\xi$ is an undetermined constant. Let $$r=\sqrt{(x_i-x_j)^2+\ldots}$$ be the distance between object $i$ and $j$. (The $y$- and $z$-coordinates are abbreviated by the ellipses.) Then $$\dot{r}=\frac{1}{r}\left[(x_i-x_j)(\dot{x}_i-\dot{x}_j)+\ldots\right]$$ and $$\ddot{r}=\frac{1}{r}\left\{\left[\left((\dot{x}_i-\dot{x}_j)^2+(x_i-x_j)(\ddot{x}_i-\ddot{x}_j)\right)+\ldots\right]-\dot{r}^2\right\}.$$ To minimize the number of floating point operations, Weber's law is taken with $G=1$, $m_i=m_j=1$, $c=1$, and $\xi=1$. This gives the acceleration of object $i$ due to object $j$ to be:  $$a_i=\frac{1}{r^2}\left[1-\frac{1}{2}\dot{r}^2+r\ddot{r}\right].$$ At time step $n$, the position $x_i$, acceleration $a_i$, and distance $r$ will be denoted $x_i(n)$, $a_i(n)$, and $r(n)$, respectively. For $N$ interacting objects, integration to future time steps proceeds as follows: $$\left[a_i(n+1)\right]_x=\sum_{i\ne j}^{N}\frac{[x_i(n)]_x-[x_j(n)]_x}{r(n)^3}\left[1-\frac{1}{2}\dot{r}(n)^2+r(n)\ddot{r}(n)\right],$$ and similarly for $y$- and $z$-components. Recall: $r$ and its derivatives are dependent on $j$. Position is computed assuming the acceleration is constant over the time step $\Delta t$; the next time step's acceleration is used: $$[x_i(n+1)]_x=[x_i(n)]_x+\sum_{i\ne j}^{N}\left(\frac{1}{2}[a_i(n+1)]_x(\Delta t)^2+[v_i(n)]_x\Delta t\right)$$ (and similarly for $y$- and $z$-components.) Lastly, velocity is updated:$$[v_i(n+1)]_x=[v_i(n)]_x+\sum_{i\ne j}^{N}[a_i(n+1)]_x\Delta t$$ (and similarly for $y$- and $z$-components). Time then updates:$$t(n+1)\rightarrow t(n)+\Delta t,$$and the preceding iterations update in the order given.
\end{document}
